%═══════════════════════════════════════════════════════════
% CHAPTER 3: MIN-MAX SPANNING TREE
% 
% Prerequisites: Ch2 (MST criteria, uncertainty definitions)
% Provides:
%   - Min-Max formulation (discrete + interval)
%   - Interval extremal lemma + proof
%   - K=2 weakly NP-hard + FULL proof (partition reduction)
%   - K=const pseudo-poly + FPTAS (cited)
%   - K unbounded strongly NP-hard + inapproximability (cited + sketched)
% Labels created:
%   - ch:minmax, sec:mm-formulation, sec:mm-extremal,
%     sec:mm-complexity, sec:mm-discussion,
%     lem:interval-extremal-cost, thm:mm-k2-hard,
%     thm:mm-kconst-pseudo, thm:mm-kunbdd-hard, fig:partition-grid
% Page budget: 10.6 pages
% Status: PLACEHOLDER
%═══════════════════════════════════════════════════════════

\chapter{Min-Max Spanning Tree}\label{ch:minmax}

% CHAPTER OVERVIEW:
% This chapter formalises the Min-Max objective for spanning trees under
% discrete scenarios and interval uncertainty. We prove the interval
% extremal lemma and provide one complete hardness proof (K=2 via partition),
% then cite stronger results for K unbounded. Complexity ranges from
% polynomial (intervals) to strongly NP-hard (discrete K unbounded).

%─────────────────────────────────────────────────────────
% SECTION 3.1: PROBLEM FORMULATION (2.2 pages)
%─────────────────────────────────────────────────────────
\section{Problem Formulation}\label{sec:mm-formulation}

% TODO: Discrete scenarios (0.9 pg)
%   - Formal objective: min_{T∈𝒯} max_{k∈[K]} c^(k)(T)
%   - Interpretation: hedge against worst scenario
%   - Worked example (micro-graph): Compute for K=3 scenarios
%   - Show T_optimal ≠ T_MST(c_mid)

% TODO: Interval uncertainty (0.9 pg)
%   - Formal objective: min_{T∈𝒯} max_{c∈∏[ℓₑ,uₑ]} c(T)
%   - Intuition: worst-case at interval boundaries (preview extremal lemma)
%   - Worked example (micro-graph): Show optimal T may differ from discrete

% TODO: Connection (0.4 pg)
%   - Discrete is special case of interval (K scenarios as vertices of box)
%   - But complexity differs dramatically (polynomial vs NP-hard)

%─────────────────────────────────────────────────────────
% SECTION 3.2: EXTREMAL PROPERTIES FOR INTERVALS (1.5 pages)
%─────────────────────────────────────────────────────────
\section{Interval Worst-Case Characterisation}\label{sec:mm-extremal}

% PROOF INVENTORY: 1 lemma + proof (0.8 pg)

% SOURCE: Implicit in Goerigk Theorem 4.8
\begin{lemma}[Interval Extremal Cost]\label{lem:interval-extremal-cost}
% TODO: Statement (0.3 pg)
% For fixed T∈𝒯, max_{c∈𝒰} c(T) is attained at c* with:
%   c*_e = u_e if e∈T, else arbitrary
\end{lemma}
\begin{proof}
% TODO: Proof (0.5 pg)
% c(T) = Σ_{e∈T} c_e is linear in c
% Maximum over product [ℓₑ,uₑ] attained at vertex (extreme point)
% For chosen edges, maximize → set to u_e
\end{proof}

% TODO: Implications (0.7 pg)
%   - Can evaluate any tree by setting chosen edges to upper bounds
%   - Min-Max with intervals = nominal MST on upper bounds → polynomial
%   - Cite: Goerigk Theorem 4.8
%   - Explain why: worst case explicit, use Kruskal/Prim on (u_e)

%─────────────────────────────────────────────────────────
% SECTION 3.3: COMPLEXITY RESULTS (3.5 pages)
%─────────────────────────────────────────────────────────
\section{Complexity and Approximation}\label{sec:mm-complexity}

\subsection{Discrete Scenarios: Constant K}

% PROOF INVENTORY: 1 full proof (1.2 pg) + 1 cited theorem (0.4 pg)

% SOURCE: Goerigk Theorem 8.4 (Kouvelis-Yu 1997)
\begin{theorem}[Partition Reduction]\label{thm:mm-k2-hard}
% TODO: Statement
% Min-Max spanning tree with K=2 discrete scenarios is weakly NP-hard.
\end{theorem}
\begin{proof}
% TODO: FULL PROOF (1.2 pg) — This is our representative hardness proof
% Reduction from PARTITION problem
% 
% SOURCE VERIFICATION:
% - Goerigk (2021), Theorem 8.4, page 245
% - Original: Kouvelis-Yu (1997)
% - Reduction: PARTITION → Min-Max ST K=2
% - Verified against /mnt/project/24_Robust_Optimization_book-1.pdf
%
% CONSTRUCTION:
% Given PARTITION instance: n items, weights w_i, target Q = (Σw_i)/2
% Build grid graph G=(V,E):
%   - Vertices: V = {v_{i,j} : i∈{1,2}, j∈{0,...,n}}
%   - Edges: E = horizontal + vertical
%   - Two scenarios: c^(1) and c^(2)
%     Scenario 1: row 1 has weights, row 2 has zeros
%     Scenario 2: row 2 has weights, row 1 has zeros
%   - All vertical edges cost zero in both scenarios
%
% EQUIVALENCE:
% PARTITION has solution ⟺ Min-Max ST has value Q
%
% (⟹) If partition P exists with Σ_{i∈P} w_i = Q:
%   Construct T using row 1 edges for i∈P, row 2 for i∉P
%   Then c^(1)(T) = Q and c^(2)(T) = Q
%
% (⟸) If optimal T* has value Q:
%   Let Q₁ = cost in scenario 1, Q₂ = cost in scenario 2
%   Since Q₁ + Q₂ = 2Q and max{Q₁,Q₂} = Q
%   Must have Q₁ = Q₂ = Q → induces partition
%
% Figure 3.1 will show grid graph structure
\end{proof}

% TODO: Figure 3.1 (0.3 pg): Grid graph construction for partition reduction
%   - TikZ figure showing 2×(n+1) grid
%   - Edge labels with scenario costs
%   - Caption: "Grid graph G for reduction from PARTITION..."
%   \label{fig:partition-grid}

% SOURCE: Goerigk Theorem 8.2-8.3 (Aissi-Bazgan-Vanderpooten 2005)
\begin{theorem}[Pseudo-Polynomial Algorithm]\label{thm:mm-kconst-pseudo}
% TODO: Statement + explanation (0.4 pg)
% For constant K, Min-Max ST admits:
%   (i) Pseudo-polynomial algorithm
%   (ii) FPTAS
\end{theorem}

% TODO: Explanation (cite + intuition, NO proof):
%   - Based on multicriteria Exact-P method
%   - Intuition: Treat as K-criteria optimisation
%   - DP on Pareto frontier → pseudo-polynomial in cost magnitudes
%   - FPTAS via scaling technique
%   - SOURCE: Goerigk Thm 8.2-8.3, Aissi-Bazgan-Vanderpooten (2005)

\subsection{Discrete Scenarios: Unbounded K}

% PROOF INVENTORY: 2 cited theorems (sketch + citation, ~1.5 pg)

% SOURCE: Goerigk Theorem 8.5 (Kasperski-Zielinski 2009)
\begin{theorem}[Strong NP-Hardness]\label{thm:mm-kunbdd-hard}
% TODO: Statement
% Min-Max ST with K part of input is strongly NP-hard and
% not approximable within (2-ε) for any ε>0.
\end{theorem}

% TODO: Sketch + citation (0.8 pg, NO full proof):
%   - Reduction from 3-SAT
%   - Gadget construction: each clause → subgraph
%   - Scenarios encode literal conflicts
%   - 3-SAT solution ⟺ Min-Max cost = 1
%   - SOURCE: Goerigk Thm 8.5, Kasperski-Zielinski (2009)
%   - Reference similar construction in Goerigk Thm 7.5 for shortest path

% TODO: Inapproximability (0.4 pg):
%   - If (2-ε)-approximable, could solve 3-SAT
%   - SOURCE: Kasperski-Zielinski (2009)

% TODO: Midpoint O(K)-approximation (0.5 pg):
%   - Midpoint heuristic: solve MST at c_mid = average over scenarios
%   - Achieves O(K) ratio
%   - SOURCE: Goerigk Corollary 5.22
%   - Open gap: O(K) vs O(log^(1-ε) n) lower bound

%─────────────────────────────────────────────────────────
% SECTION 3.4: DISCUSSION AND CONTRAST (2.1 pages)
%─────────────────────────────────────────────────────────
\section{Discussion}\label{sec:mm-discussion}

% TODO: Discrete vs Interval comparison table (0.7 pg)
%   | Model | K=const | K unbounded | Interval |
%   | Min-Max | Weak NP-hard, FPTAS | Strong NP-hard, O(K)-approx | Polynomial |

% TODO: Why intervals are easy (0.5 pg)
%   - Extremal lemma makes worst case explicit (Lemma 3.1)
%   - No inter-edge dependencies in product structure
%   - Contrast: discrete requires K worst-case evaluations

% TODO: Comparison with nominal MST (0.5 pg)
%   - Min-Max solution avoids high-upper-bound edges
%   - Micro-graph example: Compare T_MM vs T_MST(c_mid)

% TODO: Modeling remarks (0.4 pg)
%   - LP reformulations exist (cite compact formulation papers)
%   - Not our focus (structural/complexity emphasis)

%─────────────────────────────────────────────────────────
% SECTION 3.5: CHAPTER SUMMARY (0.3 pages)
%─────────────────────────────────────────────────────────
\section*{Summary}

% TODO: Summary paragraph (0.3 pg)
% Min-Max ST complexity depends critically on uncertainty structure.
% Intervals are polynomial (Lemma 3.1 + Goerigk Thm 4.8) via extremal
% evaluation. Discrete scenarios harder: weakly NP-hard even K=2
% (Theorem 3.2, partition reduction), with FPTAS for constant K
% (Theorem 3.3). For unbounded K: strongly NP-hard (Theorem 3.4),
% O(K) approximation, no constant-factor possible (Theorem 3.5).
% Chapter 5 synthesis table compares with Min-Max Regret results.

% END OF CHAPTER 3