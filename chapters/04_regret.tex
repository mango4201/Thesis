%═══════════════════════════════════════════════════════════
% CHAPTER 4: MIN-MAX REGRET SPANNING TREE
% 
% Prerequisites: Ch2 (MST), Ch3 (Min-Max for contrast)
% Provides:
%   - Regret definition and formulation
%   - Interval extremal regret lemma + proof
%   - K=2 weakly NP-hard (reuse Ch3 construction)
%   - 2-approximation for interval regret + FULL proof
% Labels created:
%   - ch:regret, sec:regret-definition, sec:regret-extremal,
%     sec:regret-complexity-discrete, sec:regret-approx-interval,
%     sec:regret-discussion,
%     lem:interval-extremal-regret, thm:regret-k2-hard,
%     thm:regret-kconst-pseudo, thm:regret-kunbdd-hard,
%     thm:regret-interval-hard, thm:regret-2approx,
%     lem:regret-lower-bound, lem:regret-upper-bound
% Page budget: 11.3 pages
% Status: PLACEHOLDER
%═══════════════════════════════════════════════════════════

\chapter{Min-Max Regret Spanning Tree}\label{ch:regret}

% CHAPTER OVERVIEW:
% This chapter formalises Min-Max Regret objective, which measures
% performance gap vs scenario-optimal solutions. Structure mirrors Ch3
% (discrete + interval), but intervals are NP-hard (unlike Min-Max).
% Positive result: clean 2-approximation via midpoint heuristic (full proof).

%─────────────────────────────────────────────────────────
% SECTION 4.1: REGRET DEFINITION AND OBJECTIVE (1.8 pages)
%─────────────────────────────────────────────────────────
\section{Regret Formulation}\label{sec:regret-definition}

% TODO: Regret concept (0.6 pg)
%   - Definition: Regret(T,c) = c(T) - MST(c)
%   - Interpretation: performance gap vs scenario-optimal
%   - Why it matters: relative vs absolute robustness

% TODO: Discrete scenarios (0.6 pg)
%   - Formal objective: min_{T∈𝒯} max_{k∈[K]} Regret(T, c^(k))
%   - Requires computing MST(c^(k)) for each scenario k
%   - Worked example (micro-graph): Show T_Regret ≠ T_MinMax

% TODO: Interval uncertainty (0.6 pg)
%   - Formal objective: min_{T∈𝒯} max_{c∈𝒰} Regret(T, c)
%   - Challenge: MST(c) varies with c in 𝒰
%   - Worked example (micro-graph): Compute regret for T₁, T₂, T₃

%─────────────────────────────────────────────────────────
% SECTION 4.2: EXTREMAL PROPERTIES FOR INTERVALS (2.2 pages)
%─────────────────────────────────────────────────────────
\section{Interval Regret Extremal Behaviour}\label{sec:regret-extremal}

% PROOF INVENTORY: 1 lemma + proof (1.0 pg)

% SOURCE: Implicit in interval regret analysis (Goerigk Ch5)
\begin{lemma}[Interval Extremal Regret]\label{lem:interval-extremal-regret}
% TODO: Statement (0.3 pg)
% For fixed T∈𝒯, max_{c∈𝒰} Regret(T,c) is attained at a
% vertex (extremal point) of the interval box ∏[ℓₑ,uₑ].
\end{lemma}
\begin{proof}
% TODO: Proof (0.7 pg)
% Regret(T,c) = c(T) - MST(c)
% c(T) is linear in c
% MST(c) is piecewise linear concave function of c
% Max of (linear - concave) over box attained at vertex
% Construction: worst case sets c_e to upper/lower strategically
% depending on which edges compete for MST
\end{proof}

% TODO: Implications (0.7 pg)
%   - Can evaluate regret of any T in polynomial time (check 2^|E| vertices)
%   - BUT: finding optimal T still NP-hard (unlike Min-Max!)
%   - Cite: Averbakh-Lebedev (2004) for NP-hardness

% TODO: Comparison with Min-Max extremal (0.5 pg)
%   - Min-Max: all chosen edges → upper (Lemma 3.1, simple)
%   - Regret: strategic mix depending on competing trees (complex)
%   - More complex but still extremal

%─────────────────────────────────────────────────────────
% SECTION 4.3: COMPLEXITY: DISCRETE SCENARIOS (2.5 pages)
%─────────────────────────────────────────────────────────
\section{Complexity for Discrete Scenarios}\label{sec:regret-complexity-discrete}

\subsection{Constant K}

% PROOF INVENTORY: 1 adapted proof (0.6 pg) + 1 cited theorem (0.4 pg)

% SOURCE: Goerigk Theorem 8.7
\begin{theorem}[K=2 Weak NP-Hardness]\label{thm:regret-k2-hard}
% TODO: Statement
% Min-Max Regret ST with K=2 discrete scenarios is weakly NP-hard.
\end{theorem}
\begin{proof}
% TODO: Proof (0.6 pg) — REUSE partition reduction from Theorem 3.2
% Same grid graph construction as Theorem 3.2 (Chapter 3)
% 
% KEY OBSERVATION:
% In that construction, MST(c^(k)) = 0 for all k=1,2
% (All vertical edges + one row's edges = zero cost tree in each scenario)
% 
% Therefore: Regret(T, c^(k)) = c^(k)(T) - 0 = c^(k)(T)
% So Min-Max Regret ≡ Min-Max for this instance
% 
% By Theorem 3.2, Min-Max is weakly NP-hard
% ∴ Min-Max Regret is weakly NP-hard
% 
% SOURCE: Goerigk Theorem 8.7 ("use same construction as Theorem 8.4")
\end{proof}

% SOURCE: Goerigk Theorem 8.6 (Aissi-Bazgan-Vanderpooten 2005)
\begin{theorem}[Pseudo-Polynomial for Constant K]\label{thm:regret-kconst-pseudo}
% TODO: Statement + explanation (0.4 pg)
% For constant K, Min-Max Regret ST admits:
%   (i) Pseudo-polynomial algorithm
%   (ii) FPTAS
\end{theorem}

% TODO: Explanation (cite + intuition, NO proof):
%   - Same multicriteria technique as Min-Max (Goerigk Thm 8.6)
%   - SOURCE: Aissi-Bazgan-Vanderpooten (2005)

\subsection{Unbounded K}

% PROOF INVENTORY: 1 cited theorem (sketch, 0.9 pg) + 1 approx (0.6 pg)

% SOURCE: Goerigk Theorem 8.8 (Kouvelis-Yu 1997)
\begin{theorem}[Strong NP-Hardness]\label{thm:regret-kunbdd-hard}
% TODO: Statement
% Min-Max Regret ST with K part of input is strongly NP-hard.
\end{theorem}

% TODO: Sketch + citation (0.9 pg, NO full proof):
%   - Reduction from 3-PARTITION problem
%   - Grid graph with K=m scenarios
%   - Construction: MST(c^(k)) = 0 in every scenario
%   - So regret reduces to cost minimisation
%   - Regret optimal ⟺ 3-PARTITION solvable
%   - SOURCE: Kouvelis-Yu (1997), Goerigk Theorem 8.8

% TODO: Midpoint O(K)-approximation (0.6 pg):
%   - Same midpoint heuristic as Min-Max
%   - Open gap (cite Goerigk Open Problem 10)

%─────────────────────────────────────────────────────────
% SECTION 4.4: APPROXIMATION FOR INTERVALS (3.3 pages)
%─────────────────────────────────────────────────────────
\section{Interval Regret Approximation}\label{sec:regret-approx-interval}

% PROOF INVENTORY: 3 theorems/lemmas with FULL proofs (2.5 pg total)

% SOURCE: Averbakh-Lebedev (2004)
\begin{theorem}[Interval NP-Hardness]\label{thm:regret-interval-hard}
% TODO: Statement + citation (0.3 pg)
% Min-Max Regret ST with interval uncertainty is NP-hard.
% SOURCE: Averbakh-Lebedev (2004)
% (Cite only, no proof — established result from network optimisation)
\end{theorem}

% SOURCE: Kasperski-Zielinski (2006), Goerigk Theorem 5.26
\begin{theorem}[2-Approximation via Midpoint]\label{thm:regret-2approx}
% TODO: Statement (0.3 pg)
% Let T_mid = MST((ℓ+u)/2). Then T_mid is a 2-approximation
% for Min-Max Regret ST with interval uncertainty.
\end{theorem}

% TODO: Supporting lemmas (prove in full before main theorem)

% SOURCE: Kasperski-Zielinski (2006), Goerigk Lemma 5.24
\begin{lemma}[Lower Bound on Regret]\label{lem:regret-lower-bound}
% TODO: Statement
% For any trees T, T' ∈ 𝒯:
%   max_{c∈𝒰} Regret(T,c) ≥ Σ_{e: xₑ>x'ₑ} ūₑ - Σ_{e: xₑ<x'ₑ} ℓₑ
% where xₑ=1 if e∈T, else 0 (similarly for T')
\end{lemma}
\begin{proof}
% TODO: Proof (0.7 pg) — FULL PROOF
% Construct adversarial scenario:
%   - For edges in T but not T': set to upper bound ū_e
%   - For edges in T' but not T: set to lower bound ℓ_e
%   - For remaining edges: set to favour T' as MST
% Then c(T) ≥ Σ_{e∈T∖T'} ū_e and MST(c) ≤ Σ_{e∈T'∖T} ℓ_e
% Regret(T,c) = c(T) - MST(c) ≥ ... (algebra)
% 
% SOURCE: Kasperski-Zielinski (2006), Goerigk Lemma 5.24
\end{proof}

% SOURCE: Kasperski-Zielinski (2006), Goerigk Lemma 5.25
\begin{lemma}[Upper Bound Relating Solutions]\label{lem:regret-upper-bound}
% TODO: Statement
% For any trees T, T' ∈ 𝒯:
%   max_{c∈𝒰} Regret(T,c) ≤ max_{c∈𝒰} Regret(T',c) + 
%                             Σ_{e: xₑ>x'ₑ} ūₑ - Σ_{e: xₑ<x'ₑ} ℓₑ
\end{lemma}
\begin{proof}
% TODO: Proof (0.7 pg) — FULL PROOF
% For any c∈𝒰:
%   Regret(T,c) = c(T) - MST(c)
%              ≤ c(T) - c(T')         (MST(c) ≥ c(T') always)
%              = Σ_{e∈T} cₑ - Σ_{e∈T'} cₑ
%              = Σ_{e∈T∖T'} cₑ - Σ_{e∈T'∖T} cₑ
% 
% Now take max over c:
%   max Regret(T,c) ≤ max [Σ_{e∈T∖T'} cₑ - Σ_{e∈T'∖T} cₑ + (Regret(T',c) - ...)]
%   ... (algebra combining worst-case scenarios)
% 
% SOURCE: Kasperski-Zielinski (2006), Goerigk Lemma 5.25
\end{proof}

\begin{proof}[Proof of Theorem~\ref{thm:regret-2approx}]
% TODO: Main proof (1.1 pg) — FULL PROOF using Lemmas 4.6-4.7
% 
% Let T_mid = MST(c_mid) where c_mid = (ℓ+u)/2
% Let T* = optimal Min-Max Regret tree
% 
% STEP 1: Use optimality of T_mid for midpoint costs:
%   Σ_{e∈T_mid} c_mid,e ≤ Σ_{e∈T*} c_mid,e
%   ⟹ Σ_{e∈T_mid} (ℓₑ+ūₑ)/2 ≤ Σ_{e∈T*} (ℓₑ+ūₑ)/2
%   Rearranging: Σ_{e∈T_mid∖T*} ūₑ - Σ_{e∈T*∖T_mid} ℓₑ ≤ 
%                Σ_{e∈T*∖T_mid} ūₑ - Σ_{e∈T_mid∖T*} ℓₑ
% 
% STEP 2: Apply Lemma 4.6 (lower bound) with T=T*, T'=T_mid:
%   max Regret(T*,c) ≥ Σ_{e: x*ₑ>x_mid,e} ūₑ - Σ_{e: x*ₑ<x_mid,e} ℓₑ
% 
% STEP 3: Apply Lemma 4.7 (upper bound) with T=T_mid, T'=T*:
%   max Regret(T_mid,c) ≤ max Regret(T*,c) + [Σ ūₑ - Σ ℓₑ]
% 
% STEP 4: Combine Step 1-3 via algebra:
%   From Step 1: symmetric difference terms relate
%   From Step 2: OPT ≥ RHS
%   From Step 3: ALG ≤ OPT + RHS
%   ⟹ ALG ≤ 2·OPT
% 
% SOURCE: Kasperski-Zielinski (2006), Goerigk Theorem 5.26
\end{proof}

% TODO: Tightness discussion (0.5 pg):
%   - Open Problem: Is 2 tight? (Goerigk Open Problem 2)
%   - No better algorithm known for any combinatorial problem
%   - No inapproximability bound either

%─────────────────────────────────────────────────────────
% SECTION 4.5: DISCUSSION AND COMPARISON (2.2 pages)
%─────────────────────────────────────────────────────────
\section{Discussion}\label{sec:regret-discussion}

% TODO: Min-Max vs Regret (0.8 pg)
%   - Conceptual: absolute cost vs relative performance
%   - Complexity parallel: both hard for discrete
%   - Intervals differ: Min-Max poly, Regret NP-hard + 2-approx
%   - Micro-graph: Show T_MinMax ≠ T_Regret for same instance

% TODO: Complexity landscape table (0.6 pg)
%   | | Min-Max Discrete | Regret Discrete | Min-Max Interval | Regret Interval |
%   | K=const | Weak NP-hard, FPTAS | Weak NP-hard, FPTAS | P | NP-hard, 2-approx |
%   | K unbdd | Strong NP-hard, O(K) | Strong NP-hard, O(K) | P | NP-hard, 2-approx |

% TODO: Why Regret harder for intervals (0.5 pg)
%   - Min-Max: extremal lemma + worst case explicit → polynomial
%   - Regret: must compare against varying MST(c) → combinatorial search
%   - 2-approximation uses midpoint as "neutral" scenario

% TODO: Modelling remarks (0.3 pg)
%   - Compact reformulations exist (dualization techniques)
%   - Branch-and-bound algorithms (Montemanni-Gambardella 2005)

%─────────────────────────────────────────────────────────
% SECTION 4.6: CHAPTER SUMMARY (0.3 pages)
%─────────────────────────────────────────────────────────
\section*{Summary}

% TODO: Summary paragraph (0.3 pg)
% Min-Max Regret complexity mirrors Min-Max for discrete scenarios
% (Theorems 4.2-4.4) but diverges for intervals. While Min-Max intervals
% are polynomial (Ch3), Regret intervals remain NP-hard (Theorem 4.8)
% despite extremal properties (Lemma 4.1). Positive result: simple
% 2-approximation via midpoint MST (Theorem 4.5, proved via Lemmas 4.6-4.7).
% This approximation is best known for any robust combinatorial problem
% with intervals, though tightness remains open. Chapter 5 synthesises
% these findings alongside Min-Max results.

% END OF CHAPTER 4