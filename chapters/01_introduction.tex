\chapter{Introduction}\label{ch:intro}

% =========================
% 1.1 Motivation and Problem Statement
% =========================
\section{Motivation and Problem Statement}\label{sec:intro-motivation}

Designing cost–efficient networks is a recurring task in transportation, telecommunications, energy, and logistics.
When edge costs are known, the \emph{minimum spanning tree} (MST) is a classical and well-understood baseline: it connects all vertices with minimum total cost.
In practice, however, costs are seldom known precisely at design time.
They depend on estimates, market prices, delays, outages, or policy changes, which may shift before the network is built or operated.
A tree that is optimal for a single nominal estimate can therefore become unexpectedly expensive when the realized costs differ.

This thesis studies spanning trees when edge costs are \emph{uncertain}.
We consider two common ways of representing uncertainty that already cover many practical situations:
(i) a \emph{discrete set of scenarios}, each a full cost vector; and
(ii) \emph{interval ranges} for each edge, forming a Cartesian product of bounds.
Given such an uncertainty description, there are (at least) two natural design goals.
The first is \emph{Min--Max}: choose a tree that keeps its \emph{worst-case total cost} as small as possible across all scenarios.
The second is \emph{Min--Max Regret}: choose a tree whose cost is, in the worst case, close to the \emph{scenario-optimal} MST; that is, we hedge not against absolute cost but against how far we fall short of the best we could have done if the scenario had been known in advance.
Both perspectives formalize robustness, yet they prioritize different risks: Min--Max is absolute (guard the total), Regret is relative (guard the performance gap).

Even at this conceptual level, two ingredients are crucial.
First, we need a clean and self-contained foundation for MST reasoning—cuts, cycles, and the fundamental exchange arguments—so that all later robust models rest on solid ground.
Second, when costs lie in intervals, we must understand \emph{where} worst cases live: Do we have to consider all points in a box, or do extremal cases suffice (e.g., bounds on selected edges)?
Answering such questions allows us to reason about robustness without enumerating infinitely many possibilities.

% --- Bridge sentence (keeps §1.1 motivational; detailed scope now lives in §1.2)
We focus on spanning trees under discrete- and interval-uncertainty and study Min--Max and Min--Max Regret objectives; precise models and notation follow in \Cref{ch:foundations}.
We also fix basic complexity and approximation vocabulary early on, enabling concise statements in later chapters.

% --- Internal notes for later tightening (comments only; do not print)
% TODO[LATER:intro-highlight]: After Chapters 3–4 are finalized, append 2–3 concrete highlight sentences
% (complexity/approximation pointers with citations) at the end of this section.
% TODO[LATER:micro-graph]: If we decide to show the running micro-graph in Ch.1, add a one-sentence
% pointer here; otherwise keep it first in Ch.2.


% =========================
% 1.2 Objectives and Scope
% =========================
\section{Objectives and Scope}\label{sec:intro-scope}

This thesis aims to provide a self-contained, correctness-first primer on robust spanning trees that carries the reader from classical MST reasoning to two robust objectives—\emph{Min--Max} and \emph{Min--Max Regret}—under two standard uncertainty styles (discrete scenarios and intervals).
The emphasis is on precise modeling, short but complete foundational proofs, and a curated synthesis of established complexity and approximation results, complemented by small worked examples.

\paragraph{Objectives.}
We pursue the following concrete objectives:
\begin{itemize}
  \item \textbf{Foundational bedrock.} Present the graph/MST basics used later (cuts, cycles, fundamental exchange arguments), with complete proofs and a tiny running example (\Cref{ch:foundations}).
    \item \textbf{Algorithmic preliminaries (complexity \& approximation).}
A concise primer (decision vs.\ optimization; P/NP; reductions; weak/strong NP-hardness; approximation ratios; PTAS/FPTAS; APX) so that results in later chapters can be stated tersely; this primer will live in Chapter~\ref{ch:foundations}.
% TODO[LATER]: once the subsection label exists, change "Chapter~\ref{ch:foundations}" to \Cref{sec:found-complexity}.
  \item \textbf{Unified modeling.} Formalize \emph{Min--Max} and \emph{Min--Max Regret} for spanning trees under (a) \emph{discrete} uncertainty and (b) \emph{interval} uncertainty, using consistent notation across chapters (\Cref{ch:minmax,ch:regret}).
  \item \textbf{Extremal reasoning for intervals.} State and use short extremal lemmas that identify where worst cases occur within interval boxes, enabling analysis without enumerating infinitely many realizations.
  \item \textbf{Curated landscape.} Collect and organize established results on computational complexity and known approximation (or inapproximability) guarantees for the above models.
  \item \textbf{Illustrative micro-examples.} Use a single small graph throughout to compare behaviors of the models and to make definitions tangible.
  \item \textbf{Synthesis and outlook.} Summarize models and results in a one-page table and outline concise takeaways and avenues for future work (\Cref{ch:synthesis,ch:conclusion}).

\end{itemize}

\paragraph{Scope and boundaries.}
We deliberately restrict attention to the following setting:
\begin{itemize}
  \item \textbf{Problem class.} Undirected, connected graphs with real-valued edge costs; solutions are spanning trees.
  \item \textbf{Uncertainty descriptions.} (i) \emph{Discrete} sets of cost scenarios, i.e., a finite set \(\Scenarios=\{c^{(1)},\dots,c^{(K)}\}\); (ii) \emph{Interval} uncertainty given by per-edge bounds forming a Cartesian product \(\prod_{e}[\,\ell_e,u_e\,]\).
  \item \textbf{Objectives.} \emph{Min--Max} (minimize worst-case total cost) and \emph{Min--Max Regret} (minimize worst-case gap to the scenario-optimal MST). Formal definitions appear in \Cref{ch:foundations} and are used in \Cref{ch:minmax,ch:regret}.
\end{itemize}

% Single pointer to avoid duplicating content from §1.4 (Roadmap).
A detailed per-chapter roadmap is given in \Cref{sec:intro-roadmap}.

% --- Internal notes for later tightening (comments only; do not print)
% TODO[LATER:technical-contrib]: After Chapters 3–4 stabilize, replace generic “curated landscape”
% phrasing with specific bullets (complexity/approximation headlines) + citations.
% TODO[LATER:appendix-choice]: If we select a representative hardness proof for Appendix A,
% add a one-sentence pointer here under Scope and update the Roadmap in §1.4 accordingly.

% =========================
% 1.3 Contributions  — deliverables & value added
% =========================
\section{Contributions}\label{sec:intro-contrib}

This section states what the reader \emph{gets} from this document—concrete deliverables and value added beyond a standard overview.
\begin{itemize}
  \item \textbf{Foundations package (Chapter~\ref{ch:foundations}).}
  A self-contained write-up of the MST toolkit (cut/cycle criteria, fundamental exchange arguments) with complete proofs, a consistent notation bank, and one reusable micro-graph figure.
  % Acceptance: proofs compile clean; figure referenced later; notation matches Appendix~B.
  
  \item \textbf{Compact complexity/approximation primer.}
A one-page glossary of the terms we use for algorithmic results, included in Chapter~\ref{ch:foundations} to keep Chapters~\ref{ch:minmax}--\ref{ch:regret} focused on model specifics.
% TODO[LATER]: swap Chapter~\ref{ch:foundations} with \Cref{sec:found-complexity} once that label exists.

  \item \textbf{Interval extremal toolkit.}
  A compact set of lemmas pinpointing where worst cases occur in interval boxes, with short proofs and guidance on how they are used later for Min--Max and Regret analyses.
  % TODO[LATER:crossref]: Insert lemma labels once fixed in Chs. 3–4.

  \item \textbf{Normalized notation and cross-referencing.}
 One consistent symbol set \((\Scenarios, \cT, \MSTcost(\cdot))\) and a clear cross-reference policy (via \texttt{\string\Cref}) applied throughout, so readers can navigate definitions/results without ambiguity.
  % Acceptance: no duplicate macros; all symbols listed in Appendix~B.

  \item \textbf{Organized literature map.}
  A tabulated map of established complexity and approximation/inapproximability results, indexed by objective (Min--Max / Regret), uncertainty (discrete / interval), and scenario parameter \(K\) where relevant.
  % TODO[LATER:results]: Replace generic wording with 2–4 concrete headline bullets + citations after Chs. 3–4.

  \item \textbf{Synthesis artifacts (Chapter~\ref{ch:synthesis}).}
  (i) A one-page summary table (models \(\times\) uncertainty \(\times\) complexity/approx); 
  (ii) a small example gallery (shared micro-graph) illustrating behaviors across objectives.
  % Acceptance: table compiles; at least two contrasting examples; consistent captions.

  \item \textbf{Optional full proof (Appendix~A).}
  One representative hardness proof reproduced in full for pedagogical completeness.
  % TODO[LATER:appendix]: Confirm the chosen problem/source; add forward reference here once selected.

  \item \textbf{Reproducible build and readable sources.}
  Clean Route~A LaTeX build (biblatex/biber, cleveref, TikZ) and a \emph{readable} Markdown export of chapters to facilitate review.
  % Acceptance: CI builds PDF; readable/*.md up-to-date on push.
\end{itemize}

% --- Internal notes for later tightening (comments only; do not print)
% Non-goals (to avoid overlap with §1.2): we do not list uncertainty styles or objectives again here.
% TODO[LATER:verify-alignment]: After Chs. 2–5 stabilize, check each bullet against actual contents and update wording.

% =========================
% 1.4 Thesis Roadmap
% =========================
\section{Thesis Roadmap}\label{sec:intro-roadmap}

This document proceeds from foundations to robust models and, finally, to a concise synthesis.

\begin{enumerate}
\item \textbf{Chapter~\ref{ch:foundations} (Foundations).}
We fix notation, prove the MST toolkit (Fundamental Cut Lemma; cycle/cut criteria), introduce a tiny running graph, and include a short primer on complexity and approximation to support the statements used in later chapters.
% TODO[LATER]: append "see \Cref{sec:found-complexity}" once the subsection is created.

  \item \textbf{Chapter~\ref{ch:minmax} (Min--Max Spanning Tree).}
  We formalize Min--Max for spanning trees under discrete scenarios and interval uncertainty, and we state the minimal interval extremal facts needed to reason about worst cases. A small worked example illustrates the model.

  \item \textbf{Chapter~\ref{ch:regret} (Min--Max Regret Spanning Tree).}
  We formalize regret and its discrete/interval variants, relate regret to scenario-optimal MSTs, and use short extremal lemmas for intervals. A worked example mirrors Chapter~\ref{ch:minmax} for direct comparison.

  \item \textbf{Chapter~\ref{ch:synthesis} (Synthesis \& Example Gallery).}
  We summarize the established landscape in a one-page table (model \(\times\) uncertainty; complexity; (in)approximability; notes on \(K\)) and gather a few TikZ micro-examples that highlight behavioral contrasts.

  \item \textbf{Chapter~\ref{ch:conclusion} (Conclusion \& Outlook).}
  We extract the main takeaways, note limitations, and briefly indicate natural extensions (e.g., budgeted-\(\Gamma\) uncertainty or polyhedral sidebars) that lie beyond our core scope.
\end{enumerate}

\paragraph{Appendices.}
Appendix~A (optional) contains one representative hardness/inapproximability proof rendered in our notation.
Appendix~B lists all symbols in a compact table for quick reference.
Appendix~C holds auxiliary figures or expanded examples referenced from the main text.

% --- Internal notes for later tightening (comments only; do not print)
% TODO[LATER:labels]: After Chapters 2–4 stabilize, add precise forward-refs to lemma/theorem labels where helpful.
% TODO[LATER:table-ref]: Insert the label for the synthesis table once finalized (e.g., \Cref{tab:synthesis}).
% TODO[LATER:examples]: If specific example numbers are introduced, append them here for quick navigation.


% (Optional) Box: Uncertainty sets at a glance (discrete, intervals; Γ-budgeted context).
