%═══════════════════════════════════════════════════════════
% CHAPTER 2: FOUNDATIONS
% 
% Prerequisites: None (self-contained)
% Provides: 
%   - Graph notation (G, V, E, cuts, cycles)
%   - Spanning tree theory (𝒯, MST criteria, 5 complete proofs)
%   - Uncertainty models (discrete, interval)
%   - Complexity glossary (P, NP, approximation)
%   - Micro-graph (Fig. 2.1)
% Labels created: 
%   - ch:foundations, sec:graph-notation, sec:mst-criteria,
%     sec:kruskal-prim, sec:uncertainty, sec:complexity,
%     lem:fund-cycle, lem:fund-cut, thm:cycle-criterion,
%     thm:cut-criterion, thm:mst-equivalence, fig:micro-graph
% Page budget: 9.3 pages
% Status: READY FOR WRITING
%═══════════════════════════════════════════════════════════

\chapter{Foundations}\label{ch:foundations}

% CHAPTER OVERVIEW:
% This chapter establishes all mathematical tools needed for Chapters 3-4.
% We prove MST optimality criteria from scratch (exchange arguments) and
% define robust optimization objectives. A fixed micro-graph is introduced
% for use in all worked examples throughout the thesis.

%─────────────────────────────────────────────────────────
% SECTION 2.1: GRAPHS, TREES, AND NOTATION (2.2 pages)
%─────────────────────────────────────────────────────────
\section{Graphs, Trees, and Notation}\label{sec:graph-notation}

% TODO: Graph model (0.4 pg)
%   - Definition: G=(V,E), undirected, connected, simple
%   - Cuts δ(X), paths, cycles
%   - Edge cost vector c ∈ ℝ^|E|

% TODO: Spanning trees (0.5 pg)
%   - Definition: T ∈ 𝒯, |E(T)| = |V|-1
%   - Tree cost: c(T) = Σ_{e∈T} cₑ
%   - MST value: MST(c) = min_{T∈𝒯} c(T)
%   - Three tree facts: unique paths, n-1 edges, removal splits

% TODO: Fundamental structures (0.5 pg)
%   - Fundamental cycle C_f for f∉T
%   - Fundamental cut δ(X_e) for e∈T

% TODO: Micro-graph figure + description (0.8 pg)
%   - TikZ figure: 4-vertex "Y-graph"
%   - Edge list with interval costs [ℓₑ, uₑ]
%     e1={1,2}: [2,8], e2={2,3}: [1,5], e3={2,4}: [3,7]
%     e4={3,4}: [2,6], e5={1,3}: [4,9]
%   - Three spanning trees: T1={e1,e2,e3}, T2={e1,e2,e4}, T3={e2,e3,e5}
%   - Three scenarios: S1 (optimistic), S2 (pessimistic), S3 (mixed)
%   - Label: \label{fig:micro-graph}
%   - Caption: "Running micro-graph G used throughout this thesis..."

%─────────────────────────────────────────────────────────
% SECTION 2.2: MST OPTIMALITY CRITERIA (3.8 pages)
%─────────────────────────────────────────────────────────
\section{Minimum Spanning Tree Optimality}\label{sec:mst-criteria}

% PROOF INVENTORY FOR THIS SECTION (3.8 pages total):
% - Lemma 2.1 (Fundamental Cycle) + proof (0.4 pg)
% - Lemma 2.2 (Fundamental Cut) + proof (0.4 pg)
% - Remark on ties (0.2 pg)
% - Theorem 2.3 (Cycle Criterion necessity) + proof (0.5 pg)
% - Theorem 2.4 (Cut Criterion necessity) + proof (0.5 pg)
% - Theorem 2.5 (Equivalence/sufficiency) + proof (0.8 pg)
% - Corollary on uniqueness (0.2 pg)
% - Micro-graph example (0.5 pg)

% TODO: Fundamental Lemmas (1.0 pg)

% SOURCE: Korte-Vygen §6.1, AMO §13.1
\begin{lemma}[Fundamental Cycle]\label{lem:fund-cycle}
% TODO: Statement
% Let T be a spanning tree and f∈E∖E(T). Then T∪{f} contains
% a unique cycle, called the fundamental cycle of f w.r.t. T.
\end{lemma}
\begin{proof}
% TODO: Proof (0.4 pg)
% Use unique path property of trees + adding edge creates cycle
\end{proof}

% SOURCE: Korte-Vygen §6.1, AMO §13.1
\begin{lemma}[Fundamental Cut]\label{lem:fund-cut}
% TODO: Statement
% Let T be a spanning tree and e∈E(T). Let X⊆V be one component
% of T\{e}. Then δ(X) contains only e among tree edges.
\end{lemma}
\begin{proof}
% TODO: Proof (0.4 pg)
% Tree removal splits into 2 components; any other tree edge would
% reconnect them → contradiction
\end{proof}

% TODO: Remark on cost ties (0.2 pg)
% Equal weights allowed, uniqueness refers to set membership

% TODO: MST Criteria Theorems (2.3 pg)

% SOURCE: Korte-Vygen Thm 6.4, AMO Thm 13.1
\begin{theorem}[MST Cycle Criterion]\label{thm:cycle-criterion}
% TODO: Statement (necessity)
% T is MST ⟹ for all e∉T, e is not strictly cheaper than max
% edge in fundamental cycle C_e
\end{theorem}
\begin{proof}
% TODO: Proof (0.5 pg)
% Assume ∃f∈C_e\{e} with c_f > c_e. Then T-f+e is cheaper → contradiction
\end{proof}

% SOURCE: Korte-Vygen Thm 6.5, AMO Thm 13.2
\begin{theorem}[MST Cut Criterion]\label{thm:cut-criterion}
% TODO: Statement (necessity)
% T is MST ⟹ for all e∈T, e is among cheapest edges in δ(X_e)
\end{theorem}
\begin{proof}
% TODO: Proof (0.5 pg)
% Assume ∃f∈δ(X_e) with c_f < c_e. Then T-e+f is cheaper → contradiction
\end{proof}

% SOURCE: Korte-Vygen Thm 6.6
\begin{theorem}[MST Characterisation]\label{thm:mst-equivalence}
% TODO: Statement (sufficiency + equivalence)
% Either criterion is necessary and sufficient for MST
\end{theorem}
\begin{proof}
% TODO: Proof (0.8 pg)
% Show cycle criterion ⟹ cut criterion via exchange arguments
% Show cut criterion ⟹ MST via greedy matroid property
\end{proof}

% TODO: Corollary on unique MST (0.2 pg)
% If all edge costs distinct, MST is unique

% TODO: Micro-graph example (0.5 pg)
% Compute MST for midpoint costs c_mid = (ℓ+u)/2
% Show fundamental cycle for one non-tree edge
% Show fundamental cut for one tree edge

%─────────────────────────────────────────────────────────
% SECTION 2.3: ALGORITHMIC ANCHORS (0.8 pages)
%─────────────────────────────────────────────────────────
\section{Kruskal and Prim: Algorithmic Remarks}\label{sec:kruskal-prim}

% TODO: Kruskal algorithm (0.4 pg)
%   - Greedy: sort edges, add if no cycle
%   - Correctness: uses cut property (Theorem 2.4)
%   - Runtime: O(m log n) with union-find
%   - NO pseudocode (remark-style only)

% TODO: Prim algorithm (0.4 pg)
%   - Greedy: grow tree from root, add cheapest leaving edge
%   - Correctness: uses cut property on {visited} vs {unvisited}
%   - Runtime: O(m log n) with priority queue
%   - NO pseudocode (remark-style only)

%─────────────────────────────────────────────────────────
% SECTION 2.4: UNCERTAINTY AND ROBUST OBJECTIVES (1.5 pages)
%─────────────────────────────────────────────────────────
\section{Uncertainty Models and Robust Optimisation}\label{sec:uncertainty}

% TODO: Discrete scenarios (0.4 pg)
%   - Definition: 𝒰 = {c^(1), ..., c^(K)}
%   - Min-Max objective: min_{T∈𝒯} max_{k∈[K]} c^(k)(T)
%   - Min-Max Regret: min_{T∈𝒯} max_{k∈[K]} [c^(k)(T) - MST(c^(k))]

% TODO: Interval uncertainty (0.4 pg)
%   - Definition: 𝒰 = ∏_{e∈E} [ℓₑ, uₑ]
%   - Min-Max objective: min_{T∈𝒯} max_{c∈𝒰} c(T)
%   - Min-Max Regret: min_{T∈𝒯} max_{c∈𝒰} [c(T) - MST(c)]

% TODO: Notation summary (0.3 pg)
%   - Scenario notation: c^(k) or \cs{k}
%   - Macro list: \Scenarios, \cT, \MSTcost{c}
%   - Pointer to Appendix A (notation table)

% TODO: Micro-graph scenarios table (0.4 pg)
%   - Scenario 1: all lower bounds c^(1) = (2,1,3,2,4)
%   - Scenario 2: all upper bounds c^(2) = (8,5,7,6,9)
%   - Scenario 3: mixed c^(3) = (5,1,7,2,4)
%   - Table: c^(k)(T₁), c^(k)(T₂), c^(k)(T₃) for k=1,2,3

%─────────────────────────────────────────────────────────
% SECTION 2.5: COMPLEXITY & APPROXIMATION GLOSSARY (1.5 pages)
%─────────────────────────────────────────────────────────
\section{Algorithmic Preliminaries}\label{sec:complexity}

% TODO: Complexity glossary (0.8 pg)
%   - Decision vs optimisation problems
%   - P, NP (verifiability), NP-hard, NP-complete
%   - Reductions (many-one, polynomial time)
%   - Weak vs strong NP-hardness
%   - Pseudo-polynomial algorithms
%   - Why K matters: K=2 vs K part of input

% TODO: Approximation glossary (0.5 pg)
%   - α-approximation (min: ALG ≤ α·OPT)
%   - PTAS, FPTAS (and why FPTAS needs pseudo-poly nominal algorithm)
%   - APX, APX-hard
%   - Inapproximability statements ("unless P=NP")

% TODO: Proof patterns (0.2 pg)
%   - Exchange arguments (used in §2.2)
%   - Extremal arguments for intervals (Ch3-4)
%   - Reductions for hardness (Ch3-4)

%─────────────────────────────────────────────────────────
% SECTION 2.6: CHAPTER SUMMARY (0.3 pages)
%─────────────────────────────────────────────────────────
\section*{Summary}

% TODO: Summary paragraph (0.3 pg)
% We have established the complete MST toolkit:
%   - Fundamental cycle/cut lemmas (Lemmas 2.1-2.2)
%   - Optimality criteria via exchange arguments (Theorems 2.3-2.5)
%   - Sections 2.4-2.5 define uncertainty models and complexity terminology
%   - Micro-graph (Fig. 2.1) will be reused in all worked examples (§3.4, §4.5, §5.2)
%   - All notation tabulated in Appendix A

% END OF CHAPTER 2