\chapter{Foundations}\label{ch:foundations}
\section{Graphs, Trees, and Notation}\label{sec:found-notation}
% G=(V,E), costs c_e, cuts \delta(X), cycles, paths, spanning trees \mathcal{T}.
% Fundamental cycle C_e for e \notin T; fundamental cut \delta(X_e) for e \in T.

\section{Fundamental Cut Lemma}\label{sec:found-fundcut}
\begin{lemma}[Fundamental Cut Lemma]\label{lem:fund-cut}
Let $T$ be a spanning tree and $e\in E(T)$. Then $\delta(X_e)\cap E(T)=\{e\}$.
\end{lemma}
\begin{proof}
% Full proof here (tree splits into two components; any other tree edge in the cut reconnects them).
\end{proof}

\section{MST Optimality: Cycle and Cut Criteria}\label{sec:found-mst-criteria}
\begin{theorem}\label{thm:mst-criteria}
For a spanning tree $T$, the following are equivalent:
(i) $T$ is an MST; (ii) every non-tree edge is a most expensive edge on its fundamental cycle;
(iii) every tree edge is among the cheapest edges in its fundamental cut.
\end{theorem}
\begin{proof}[Proof sketch]
% Standard exchange arguments (cite textbooks).
\end{proof}

\section{Kruskal and Prim (Brief Recap)}\label{sec:found-kruskal-prim}
% One paragraph each; correctness references \Cref{thm:mst-criteria}.

\section{Uncertainty Sets and Robust Criteria (Definitions Only)}\label{sec:found-robust-defs}
% Discrete scenarios \mathcal{U}, interval bounds; define Min-Max and Min-Max Regret objectives formally.

% =========================
% Algorithmic Preliminaries (Complexity & Approximation)
% =========================
\section{Algorithmic Preliminaries: Complexity and Approximation}\label{sec:found-complexity}

\subsection{Decision vs.\ Optimization; P, NP, Reductions}\label{sec:found-pnp}
% TODO: Briefly separate decision vs. optimization versions.
% TODO: Define polynomial time, class P, class NP (verifiability).
% TODO: Define reductions (many-one), NP-hard, NP-complete.
% TODO: Weakly vs. strongly NP-hard; pseudo-polynomial algorithms.
% TODO: Cite concise sources (OptiB Ch.~10; TGNO §1.1 bullets; AMO Ch.~3). 
% Refs: \cite{Goerigk2021RCO} for robust context pointers. 
% Cross-refs: Used explicitly in \Cref{sec:mm-complexity,sec:regret-complexity}.

\subsection{Approximation Notions}\label{sec:found-approx}
% TODO: Define \alpha-approximation (minimization), approximation ratio.
% TODO: PTAS vs. FPTAS; why only FPTAS implies pseudo-polytime.
% TODO: APX-hard / inapproximability statement form (“unless P=NP”).
% TODO: Mention K-dependent results: how scenario count K enters ratios/complexity.
% Cross-refs: Will be invoked in \Cref{sub:mm-k,sub:regret-k,sub:regret-interval-hard}.

\subsection{Proof Patterns We Will Use}\label{sec:found-patterns}
% TODO: (1) Exchange arguments for MST (already in §\ref{sec:found-mst-criteria}).
% TODO: (2) Extremal arguments on boxes (interval sets) for fixed solutions.
% TODO: (3) Gap-preserving reductions (only at a high level; details deferred to App.~\ref{app:rep-proof} if used).
% Cross-refs: \Cref{lem:mm-extremal-cost,lem:regret-extremal} later.

% --- Optional: tiny checklist for readers
\paragraph{Reader’s checklist (non-printing comments).}
% After reading §\ref{sec:found-complexity}, the reader should be able to:
% - parse statements like “NP-hard even for unbounded K”; 
% - understand statements like “admits a 2-approximation under intervals”.


\section{Running Micro-Graph and Notation Table}\label{sec:found-micro}
% Include small TikZ; if big symbol table, move to App. B and reference here.
