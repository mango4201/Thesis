%═══════════════════════════════════════════════════════════
% CHAPTER 5: SYNTHESIS AND EXAMPLE GALLERY
% 
% Prerequisites: Ch2-4 (all results established)
% Provides:
%   - Comprehensive results table (2 objectives × 2 models × 2 K cases)
%   - Key patterns synthesis
%   - TikZ example gallery (micro-graph under 4 objectives)
%   - Extremal behavior visualisation
%   - Takeaways for practitioners
% Labels created:
%   - ch:synthesis, sec:results-table, sec:example-gallery,
%     sec:takeaways,
%     tab:complexity-landscape, fig:micro-graph-gallery,
%     fig:extremal-visualisation
% Page budget: 6.3 pages
% Status: PLACEHOLDER
%═══════════════════════════════════════════════════════════

\chapter{Synthesis and Example Gallery}\label{ch:synthesis}

% CHAPTER OVERVIEW:
% This chapter consolidates all complexity and approximation results from
% Chapters 3-4 into a unified framework. A comprehensive table classifies
% results by objective and uncertainty model. The micro-graph example
% gallery illustrates that optimal solutions differ across models. We
% extract key takeaways for robust spanning tree optimization.

%─────────────────────────────────────────────────────────
% SECTION 5.1: SUMMARY OF RESULTS (2.5 pages)
%─────────────────────────────────────────────────────────
\section{Comprehensive Results Classification}\label{sec:results-table}

\subsection{Complexity and Approximation Landscape}

% TODO: Comprehensive table (1.5 pg) — This is Table 5.1

% Table 5.1: Complexity and approximation landscape for robust spanning trees
% 
% | Objective | Uncertainty | K=constant | K unbounded | Key Observations |
% |-----------|------------|------------|-------------|------------------|
% | Min-Max | Discrete | Weakly NP-hard (K=2, Thm 3.2) | Strongly NP-hard (Thm 3.4) | Extremal: max at scenarios |
% |         |          | Pseudo-poly, FPTAS (Thm 3.3) | O(K)-approx (midpoint) | Complexity grows with K |
% |         |          |                               | Not (2-ε)-approx (Thm 3.5) | |
% | Min-Max | Interval | POLYNOMIAL (Lemma 3.1) | POLYNOMIAL | Extremal: chosen edges → upper |
% |         |          | MST on upper bounds    |            | Reduces to nominal MST |
% | Regret  | Discrete | Weakly NP-hard (K=2, Thm 4.2) | Strongly NP-hard (Thm 4.4) | Reuses Min-Max constructions |
% |         |          | Pseudo-poly, FPTAS (Thm 4.3)  | O(K)-approx (midpoint)      | MST(c^k) complicates but |
% |         |          |                                |                             | doesn't change complexity |
% | Regret  | Interval | NP-hard (Thm 4.8, AL04) | NP-hard | Extremal: strategic boundaries |
% |         |          | 2-approximation (Thm 4.5) | 2-approximation | MST(c) varies → harder than Min-Max |
% |         |          | via midpoint              | via midpoint    | Best known approx for any problem! |
%
% \label{tab:complexity-landscape}
% Caption: "Complete complexity and approximation classification..."

\subsection{Key Patterns and Insights}

% TODO: Pattern synthesis (1.0 pg)

% PATTERN 1: Extremal behaviour
%   - All interval objectives have worst cases at boundaries
%   - Min-Max: simple (all chosen → upper)
%   - Regret: complex (strategic mix)

% PATTERN 2: K matters for discrete
%   - Constant K → pseudo-poly + FPTAS (multicriteria method)
%   - Unbounded K → strong NP-hard (3-SAT/3-PARTITION reductions)
%   - Threshold at K=2 already weakly NP-hard

% PATTERN 3: Interval dichotomy
%   - Min-Max easy (polynomial via extremal evaluation)
%   - Regret hard (NP-hard, but 2-approx available)
%   - Fundamental difference: fixed MST vs varying MST(c)

% PATTERN 4: Midpoint heuristic universality
%   - Works for both objectives (Min-Max and Regret)
%   - Performance: O(K) for discrete, 2-exact for Regret intervals
%   - Simple practical algorithm despite theoretical limits

% PATTERN 5: Open questions
%   - Tightness of 2-approximation (Goerigk Open Problem 2)
%   - Gap between O(K) and O(log^(1-ε) n) (Open Problems 10, 12)

%─────────────────────────────────────────────────────────
% SECTION 5.2: ILLUSTRATIVE EXAMPLE GALLERY (3.0 pages)
%─────────────────────────────────────────────────────────
\section{Example Gallery}\label{sec:example-gallery}

\subsection{Micro-Graph Solutions Across Objectives}

% TODO: Figure 5.1 (1.5 pg) — TikZ gallery with 4 subfigures
% 
% Figure 5.1: Micro-graph solutions under different objectives
% 
% (a) Nominal MST(c_mid):
%     - Solve MST with midpoint costs c_mid = (ℓ+u)/2
%     - Highlight optimal tree T_nominal in red
%     - Show cost: MST(c_mid) = ...
% 
% (b) Min-Max discrete (K=3 scenarios):
%     - Optimal tree T_MM_discrete (may differ from T_nominal)
%     - Show worst-case cost: max_{k∈{1,2,3}} c^(k)(T_MM_discrete) = ...
% 
% (c) Min-Max interval:
%     - Optimal tree T_MM_interval = MST(upper bounds)
%     - Show worst-case cost: max_{c∈𝒰} c(T_MM_interval) = ...
% 
% (d) Regret interval:
%     - Optimal or near-optimal tree T_Regret
%     - Show worst-case regret: max_{c∈𝒰} Regret(T_Regret, c) = ...
% 
% Table below figure:
% | Tree | c_mid | Max cost (discrete) | Max cost (interval) | Max regret (interval) |
% |------|-------|---------------------|---------------------|-----------------------|
% | T₁   | ...   | ...                 | ...                 | ...                   |
% | T₂   | ...   | ...                 | ...                 | ...                   |
% | T₃   | ...   | ...                 | ...                 | ...                   |
% 
% \label{fig:micro-graph-gallery}
% Caption: "Comparison of optimal solutions for the micro-graph (Fig. 2.1)
%           under different robust objectives. Highlighted edges form the
%           optimal tree for each model. Observe that optimal solutions
%           differ across objectives, validating the need for robust optimisation."

\subsection{Extremal Behaviour Visualisation}

% TODO: Figure 5.2 (0.8 pg) — TikZ geometric interpretation
% 
% Figure 5.2: Geometric interpretation of extremal lemmas
% 
% Illustrate interval box [ℓ,u]^|E| as:
%   - 2D or 3D cube for visualisation
%   - Vertices (extremal points) marked
%   - For Min-Max: show vertex where chosen edges at upper bounds
%   - For Regret: show vertex with strategic boundary assignment
% 
% \label{fig:extremal-visualisation}
% Caption: "Geometric interpretation of extremal lemmas (Lemmas 3.1, 4.1).
%           For fixed tree T, worst-case cost/regret lives at vertices
%           (extremal points) of the interval uncertainty box. Min-Max uses
%           simple rule (chosen → upper), Regret uses strategic assignment."

\subsection{Worked Example: Partition Reduction}

% TODO: Small partition example (0.7 pg)
% 
% Instance: PARTITION problem with items {3,4,5,6}, target Q=9
% 
% Grid graph construction (reference Figure 3.1):
%   - Show 2×5 grid (n=4 items)
%   - Edge costs in scenarios 1 and 2
%   - Solution tree: row 1 uses {3,6}, row 2 uses {4,5}
%   - Both scenarios have cost Q=9
% 
% Illustrates: How hardness proof works (Theorem 3.2)

%─────────────────────────────────────────────────────────
% SECTION 5.3: KEY TAKEAWAYS (0.5 pages)
%─────────────────────────────────────────────────────────
\section{Key Takeaways}\label{sec:takeaways}

% TODO: 5-6 bullet synthesis (0.5 pg)

% TAKEAWAY 1: Interval extremal principle
%   Worst cases always occur at interval boundaries for both Min-Max
%   and Regret (Lemmas 3.1, 4.1). This geometric insight enables
%   tractable evaluation of any candidate solution.

% TAKEAWAY 2: Complexity hierarchy
%   Polynomial nominal MST → NP-hard robust variants. Progression:
%   Min-Max intervals (poly) → Min-Max discrete K=const (weak NP-hard, FPTAS) →
%   Min-Max discrete K unbounded (strong NP-hard) → Regret intervals (NP-hard, 2-approx).

% TAKEAWAY 3: K is critical parameter
%   Constant K allows FPTAS via multicriteria methods (Theorems 3.3, 4.3).
%   Unbounded K yields strong NP-hardness via SAT/partition reductions
%   (Theorems 3.4, 4.4). Threshold behaviour at K=2.

% TAKEAWAY 4: Approximation asymmetry
%   Min-Max discrete: hard to approximate beyond O(K), no constant-factor
%   possible (Theorem 3.5). Regret intervals: clean 2-approximation exists
%   (Theorem 4.5), though tightness unknown.

% TAKEAWAY 5: Midpoint heuristic universality
%   Simple practical algorithm: solve MST at midpoint/average costs.
%   Works across both objectives with provable guarantees (O(K) or 2-exact).
%   Despite theoretical limits, often best choice in practice.

% TAKEAWAY 6: Open frontier
%   Several approximation gaps remain open (Goerigk Ch11):
%   (i) Is 2-approx tight for interval regret? (Open Problem 2)
%   (ii) Close gap O(K) vs O(log^(1-ε) n) for discrete problems? (Open Problems 10, 12)

%─────────────────────────────────────────────────────────
% SECTION 5.4: CHAPTER SUMMARY (0.3 pages)
%─────────────────────────────────────────────────────────
\section*{Summary}

% TODO: Summary paragraph (0.3 pg)
% Table 5.1 synthesises all complexity and approximation results from
% Chapters 3-4, indexed by objective and uncertainty structure. The
% micro-graph example gallery (Figures 5.1-5.2) illustrates that optimal
% solutions differ across models, validating the need for robust optimisation.
% The extremal lemmas (Lemmas 3.1, 4.1) have geometric interpretation:
% worst cases live at vertices of the interval box. The 2-approximation
% for interval regret (Theorem 4.5) stands out as the best known result
% for any robust combinatorial problem. Several approximation gaps remain
% open challenges for future research (§5.3).

% END OF CHAPTER 5